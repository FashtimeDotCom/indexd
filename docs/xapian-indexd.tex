% -*- mode: LaTex; -*-
\documentclass[unicode]{beamer}
\usepackage{hyperref}
\usepackage{color}
\usepackage{CJKutf8}
\usepackage{booktabs}
\usepackage{multirow}
\usepackage{tikz}

\usetikzlibrary{calc,decorations.pathmorphing,patterns}

%\usetheme{Berkeley}
%\usetheme{Madrid}
\usetheme{Rochester}
%\usetheme{CambridgeUS}
%\usetheme{Antibes}

\def\docver{{\tiny v1}}

\title{Xapian and Indexd}
\subtitle{A search service based on Xapian \docver}
\author{KDr2}

\begin{document}
\begin{CJK}{UTF8}{gbsn}

  \maketitle{}
  \date

  \section{大纲}
  \begin{frame}
    \frametitle{大纲}
    \setcounter{tocdepth}{2}
    \tableofcontents
  \end{frame}

  \section{Xapian Glossary}
  \begin{frame}
    \frametitle{Xapian Glossary}
    \begin{columns}[t]
      \begin{column}{0.4\textwidth}
        Indexing Concepts:\vfill{}
        \begin{itemize}
        \item Database \vfill
        \item Document \newline
          {\tiny term/value/data} \vfill
        \item Term \vfill
        \item Position \vfill
        \item Posting \vfill
        \item TermGenerator \vfill
        \item Stemer \vfill
        \item Value \vfill
        \item Prefix \vfill
        \end{itemize}
      \end{column}
      \begin{column}{0.4\textwidth}
        Query Concepts:\vfill{}
        \begin{itemize}
        \item Queries \vfill
          \begin{itemize}
          \item simple queries
          \item logic operators \newline
            {\tiny (OR/AND/AND\_NOT)}
          \item Maybe {\tiny (AND\_MAYBE)}
          \item Near and Phrase
          \end{itemize}
        \item QueryParser \vfill
          \begin{itemize}
          \item Prefix, Field:Value
          \end{itemize}
        \end{itemize}
      \end{column}
      \end{columns}
  \end{frame}

  \begin{frame}
    \frametitle{Some Documents for Xapian}
    \begin{itemize}
    \item Getting Started with Xapian\\
      {\tiny \url{http://getting-started-with-xapian.readthedocs.org/en/latest/index.html}}
    \item Xapian API Doc (generated by Doxygen) \\
      {\tiny \url{http://xapian.org/docs/apidoc/html/index.html}}
    \item Omega: an application built upon Xapian \\
      {\tiny \url{http://xapian.org/docs/omega/}}
    \item Term Prefixes:\\
      {\tiny \url{http://xapian.org/docs/omega/termprefixes.html}}
    \end{itemize}
  \end{frame}

  \section{Indexd}
  \begin{frame}
    \frametitle{Indexd}
    The Implementation: \vfill
    \begin{itemize}
    \item Run as a Service \vfill
    \item AWTP02/Redis-Compatible \vfill
    \item Gevent-Based \vfill
    \item use SCWS as TermGenerator \vfill
    \item Multi Databases Support \vfill
    \item Map docid and model-id (using Document Data)
    \end{itemize}
  \end{frame}

  \begin{frame}
    \frametitle{Next to do?}
    Do these works if you take interest: \vfill
    \begin{itemize}
    \item A Real Stemer for Chinese
    \item A Better TermGenerator
    \item Complex Queries support
    \item Redis-Compitable Protocol
    \item TestCases, Benchmarks
    \item $\cdots$
    \end{itemize}
  \end{frame}

  \section{Indexd in Haddit}
  \begin{frame}
    \frametitle{Indexd in Haddit}
    \begin{itemize}
    \item Indexing
      \begin{itemize}
      \item model-create/update/delete
      \item `indexing' table in TypeTroller \\
        {\tiny what to indexing/storing/ranking}
      \item importing old data
      \item $\cdots$
      \end{itemize}
      \vfill
    \item Query
      \begin{itemize}
      \item AWTP02 client
      \item SearchOperation for Meddit
      \item Parse HTTP-Request to Query
      \item $\cdots$
      \end{itemize}
    \end{itemize}
  \end{frame}

  \section{结束}
  \begin{frame}
    \begin{center}
      {\tiny 有兴趣的可根据此提纲自己学习,如有需要可聚众讨论}\\\vfill
      {\Huge END}
    \end{center}
  \end{frame}

\end{CJK}
\end{document}
